% Feedback Collection & Continuous Improvement Section

% Effective Feedback Mechanisms
\begin{frame}
    \frametitle{User Feedback: Binary Feedback}
    
    \begin{center}
        \begin{tikzpicture}
            \node[draw, rounded corners, fill=blue!5, text width=9cm, align=center, padding=0.3cm] {
                \textbf{Simple Binary Feedback with Categories}
            };
        \end{tikzpicture}
    \end{center}
    
    \begin{columns}
        \column{0.48\textwidth}
        \textbf{Feedback UI}
        \begin{itemize}
            \item Thumbs up/down buttons
            \item Simple and prominent
            \item Quick to complete
        \end{itemize}
        
        \column{0.48\textwidth}
        \textbf{Reason Categories}
        \begin{itemize}
            \item Irrelevant results
            \item Incomplete answer
            \item Incorrect information
            \item Outdated content
        \end{itemize}
    \end{columns}
    
    \begin{center}
        \begin{tikzpicture}
            \node[draw, rounded corners, fill=green!5, text width=8.5cm, align=center, padding=0.3cm] {
                \textbf{Key insight:} Even simple feedback is better than none
            };
        \end{tikzpicture}
    \end{center}
\end{frame}

\begin{frame}
    \frametitle{User Feedback: Citation \& Implicit Signals}
    
    \begin{columns}
        \column{0.48\textwidth}
        \textbf{Citation Validation}
        \begin{itemize}
            \item "Were these sources helpful?"
            \item Mark irrelevant citations
            \item Track valuable sources
            \item Identify missing citations
        \end{itemize}
        
        \column{0.48\textwidth}
        \textbf{Implicit Signals}
        \begin{itemize}
            \item Time spent reviewing results
            \item Follow-up questions
            \item Copied/saved content
            \item Repeated queries
        \end{itemize}
    \end{columns}
    
    \begin{center}
        \begin{tikzpicture}
            \node[draw, rounded corners, fill=yellow!10, text width=8.5cm, align=center, padding=0.3cm] {
                \textbf{Pro tip:} Combine explicit and implicit signals for a complete picture
            };
        \end{tikzpicture}
    \end{center}
\end{frame}

% Feedback Implementation Tips
\begin{frame}
    \frametitle{Feedback UI Design}
    
    \begin{columns}
        \column{0.48\textwidth}
        \textbf{UI Elements}
        \begin{itemize}
            \item Large, obvious buttons
            \item Modal dialogs for higher engagement
            \item 1-2 clicks for initial feedback
            \item Optional detailed feedback
        \end{itemize}
        
        \column{0.48\textwidth}
        \textbf{Feedback Categories}
        \begin{itemize}
            \item Retrieval vs. generation issues
            \item Missing vs. incorrect info
            \item Technical errors vs. content gaps
            \item UI/UX problems
        \end{itemize}
    \end{columns}
    
    \begin{center}
        \begin{tikzpicture}
            \node[draw, rounded corners, fill=blue!10, text width=8.5cm, align=center, padding=0.3cm] {
                \textbf{Did you know?} Modal dialogs get 4-5x more feedback than subtle buttons
            };
        \end{tikzpicture}
    \end{center}
\end{frame}

\begin{frame}
    \frametitle{Feedback Storage \& Analysis}
    
    \textbf{Store Complete Context}
    \begin{itemize}
        \item Original query and retrieved documents
        \item Generated response and citations
        \item User feedback and follow-up actions
        \item System metadata (latency, model version, etc.)
    \end{itemize}
    
    \vspace{0.3cm}
    \begin{center}
        \begin{tikzpicture}
            \node[draw, rounded corners, fill=green!5, text width=8.5cm, align=center, padding=0.3cm] {
                \textbf{Example schema:}\\
                \texttt{\{query, results, response, feedback, metadata\}}
            };
        \end{tikzpicture}
    \end{center}
\end{frame}

% Continuous Improvement Cycle
\begin{frame}
    \frametitle{Feedback Analysis Cycle}
    
    \begin{columns}
        \column{0.48\textwidth}
        \textbf{Weekly Review}
        \begin{itemize}
            \item Analyze feedback patterns
            \item Track trends over time
            \item Identify common failures
            \item Share insights with team
        \end{itemize}
        
        \column{0.48\textwidth}
        \textbf{Prioritization}
        \begin{itemize}
            \item Focus on high-volume issues
            \item Balance quick wins vs. structural fixes
            \item Create targeted test cases
            \item Track impact of fixes
        \end{itemize}
    \end{columns}
\end{frame}

\begin{frame}
    \frametitle{Continuous Improvement Process}
    
    \textbf{Creating Test Cases}
    \begin{itemize}
        \item Convert real failures into test examples
        \item Expand test coverage based on user behavior
        \item Validate fixes against expanded test set
    \end{itemize}
    
    \vspace{0.3cm}
    \textbf{Model Refinement}
    \begin{itemize}
        \item Use feedback to create training pairs
        \item Fine-tune embeddings or rerankers
        \item Improve router accuracy with real examples
    \end{itemize}
    
    \begin{center}
        \begin{tikzpicture}
            \node[draw, rounded corners, fill=yellow!10, text width=8.5cm, align=center, padding=0.3cm] {
                \textbf{Key milestone:} When feedback drives automatic system improvements
            };
        \end{tikzpicture}
    \end{center}
\end{frame}

% Case Study: Rejected URLs in Sales Follow-Up
\begin{frame}
    \frametitle{Case Study: Rejected URLs in Sales Follow-Up}
    
    \begin{columns}
        \column{0.48\textwidth}
        \textbf{The Challenge}
        \begin{itemize}
            \item System wrote post-call follow-up emails with links
            \item AI kept hallucinating or mistyping certain URLs
            \item 4\% error rate on links in emails
            \item Damaged credibility with customers
        \end{itemize}
        
        \column{0.48\textwidth}
        \textbf{The Solution}
        \begin{itemize}
            \item Introduced URL validator
            \item Rejected links to non-existent pages or unknown domains
            \item Asked model to remove/fix invalid links
            \item Later fine-tuned model to avoid invalid links
        \end{itemize}
    \end{columns}
    
    \vspace{0.3cm}
    \begin{center}
        \begin{tikzpicture}
            \node[draw, rounded corners, fill=green!10, text width=9cm, align=center, padding=0.3cm] {
                \textbf{Key Takeaway:} Simple post-processing + feedback loop reduced errors to nearly 0\%
            };
        \end{tikzpicture}
    \end{center}
\end{frame}

% Case Study: Changing Copy to Collect More Feedback
\begin{frame}
    \frametitle{Case Study: Changing Copy to Collect More Feedback}
    
    \begin{columns}
        \column{0.48\textwidth}
        \textbf{The Challenge}
        \begin{itemize}
            \item Team wanted more thumbs up/down data
            \item Tiny button labeled "How did we do?" hidden at top
            \item Very few users provided ratings
            \item Insufficient data for measuring correctness
        \end{itemize}
        
        \column{0.48\textwidth}
        \textbf{The Solution}
        \begin{itemize}
            \item Changed copy to "Did we answer your question?"
            \item Made feedback UI larger and more central
            \item Requested brief reason for thumbs down
            \item Categorized feedback for targeted improvements
        \end{itemize}
    \end{columns}
    
    \vspace{0.3cm}
    \begin{center}
        \begin{tikzpicture}
            \node[draw, rounded corners, fill=blue!10, text width=9cm, align=center, padding=0.3cm] {
                \textbf{Key Takeaway:} UI changes increased feedback volume 4-5x, providing crucial data for improvements
            };
        \end{tikzpicture}
    \end{center}
\end{frame}

% Streaming Responses - New section based on Week 6 content
\begin{frame}
    \frametitle{Streaming Responses: Improving Perceived Performance}
    
    \begin{columns}
        \column{0.48\textwidth}
        \textbf{Benefits of Streaming}
        \begin{itemize}
            \item Reduces perceived latency by ~11\%
            \item Users tolerate longer wait times
            \item Allows immediate reading while generation continues
            \item Provides visual feedback on progress
        \end{itemize}
        
        \column{0.48\textwidth}
        \textbf{Implementation Approaches}
        \begin{itemize}
            \item Stream tokens as they're generated
            \item Show interstitials during processing steps
            \item Render skeleton screens during loading
            \item Display tool execution in real-time
        \end{itemize}
    \end{columns}
    
    \begin{center}
        \begin{tikzpicture}
            \node[draw, rounded corners, fill=yellow!10, text width=9cm, align=center, padding=0.3cm] {
                \textbf{Did you know?} Users will wait up to 8 seconds longer when given visual feedback
            };
        \end{tikzpicture}
    \end{center}
\end{frame}

\begin{frame}
    \frametitle{Streaming Implementation: Interstitials \& UI}
    
    \begin{itemize}
        \item \textbf{Interstitial messages} during processing steps:
        \begin{itemize}
            \item "Thinking..." → "Searching documents..." → "Reading results..." → "Generating response..."
        \end{itemize}
        \item \textbf{Structured streaming} for complex responses:
        \begin{itemize}
            \item Stream partial JSON objects with content, citations, follow-ups
            \item Parse and render components as they arrive
        \end{itemize}
        \item \textbf{Tool execution visualization}:
        \begin{itemize}
            \item Show function calls and parameters in real-time
            \item Allow user edits to parameters before execution
        \end{itemize}
    \end{itemize}
    
    \begin{center}
        \begin{tikzpicture}
            \node[draw, rounded corners, fill=green!5, text width=9cm, align=center, padding=0.3cm] {
                \textbf{Key insight:} Streaming isn't just about tokens—it's about communicating process
            };
        \end{tikzpicture}
    \end{center}
\end{frame}

% Rejecting Work - New section based on Week 6 content
\begin{frame}
    \frametitle{Rejecting Work: Setting Expectations}
    
    \begin{columns}
        \column{0.48\textwidth}
        \textbf{The Problem}
        \begin{itemize}
            \item Not all queries can be answered well
            \item 90\% success rate means 10\% failures
            \item Attempting all queries reduces trust
            \item Users blame system for poor answers
        \end{itemize}
        
        \column{0.48\textwidth}
        \textbf{The Solution}
        \begin{itemize}
            \item Identify query types with low success
            \item Gracefully reject answering these
            \item Set clear expectations with users
            \item Collect feedback on rejected queries
        \end{itemize}
    \end{columns}
    
    \begin{center}
        \begin{tikzpicture}
            \node[draw, rounded corners, fill=yellow!10, text width=9cm, align=center, padding=0.3cm] {
                \textbf{Example:} "I can't confidently answer this question with the information available. Would you like me to try anyway with the understanding that the answer may be incomplete?"
            };
        \end{tikzpicture}
    \end{center}
\end{frame}

\begin{frame}
    \frametitle{Showcasing Capabilities}
    
    \begin{itemize}
        \item \textbf{Highlight what your system does well}:
        \begin{itemize}
            \item Suggest example queries users can try
            \item Demonstrate different capabilities through UI
            \item Show different content types you can handle
        \end{itemize}
        \item \textbf{Implementation approaches}:
        \begin{itemize}
            \item Categorized example queries on landing page
            \item "Focus" buttons to expand capability options
            \item Follow-up suggestions after each response
            \item Special UI components for different content types
        \end{itemize}
    \end{itemize}
    
    \begin{center}
        \begin{tikzpicture}
            \node[draw, rounded corners, fill=green!5, text width=9cm, align=center, padding=0.3cm] {
                \textbf{Key insight:} Users will use the capabilities you highlight and ignore those you don't
            };
        \end{tikzpicture}
    \end{center}
\end{frame}

% Chain of Thought and Monologuing - New section based on Week 6 content
\begin{frame}
    \frametitle{Chain of Thought: Improving Reasoning}
    
    \begin{columns}
        \column{0.48\textwidth}
        \textbf{Benefits}
        \begin{itemize}
            \item ~10\% performance improvement
            \item Makes complex reasoning possible
            \item Provides transparency into model thinking
            \item Enables better error analysis
        \end{itemize}
        
        \column{0.48\textwidth}
        \textbf{Implementation}
        \begin{itemize}
            \item Structure as XML components
            \item Stream thinking process separately
            \item Allow users to expand/collapse
            \item Collect feedback on reasoning errors
        \end{itemize}
    \end{columns}
    
    \begin{center}
        \begin{tikzpicture}
            \node[draw, rounded corners, fill=yellow!10, text width=9cm, align=center, padding=0.3cm] {
                \textbf{Key insight:} Chain of thought can be the difference between usable and unusable
            };
        \end{tikzpicture}
    \end{center}
\end{frame}

\begin{frame}
    \frametitle{Monologuing: Managing Complex Contexts}
    
    \begin{itemize}
        \item \textbf{Reiterate relevant information} before generating responses:
        \begin{itemize}
            \item Restate key instructions from the prompt
            \item Summarize relevant parts of the context
            \item Co-locate related information for better attention
        \end{itemize}
        \item \textbf{Multi-stage monologuing} for complex tasks:
        \begin{itemize}
            \item Identify relevant variables first
            \item Extract relevant sections from documents
            \item Connect information across sources
            \item Reason about options before generating final response
        \end{itemize}
    \end{itemize}
    
    \begin{center}
        \begin{tikzpicture}
            \node[draw, rounded corners, fill=green!5, text width=9cm, align=center, padding=0.3cm] {
                \textbf{Case study:} SaaS pricing quotes improved dramatically with 4-stage monologuing
            };
        \end{tikzpicture}
    \end{center}
\end{frame}

% Key Insights for Week 6
\begin{frame}
    \frametitle{Week 6: Key Insights}
    
    \begin{center}
        \begin{tikzpicture}
            \node[draw, rounded corners, fill=blue!5, text width=9cm, align=center, padding=0.3cm] {
                \textbf{Product Considerations: UX, Feedback \& Prompting}
            };
        \end{tikzpicture}
    \end{center}
    
    \begin{itemize}
        \item \textbf{Design prominent feedback mechanisms}
        \begin{itemize}
            \item Make feedback UI large and obvious
            \item Use modal dialogs for higher engagement
        \end{itemize}
        \item \textbf{Implement streaming responses}
        \begin{itemize}
            \item Reduce perceived latency with visual feedback
            \item Use interstitials to communicate processing steps
        \end{itemize}
        \item \textbf{Know when to reject work}
        \begin{itemize}
            \item Set clear expectations for low-confidence answers
            \item Showcase capabilities you excel at
        \end{itemize}
        \item \textbf{Leverage chain of thought \& monologuing}
        \begin{itemize}
            \item Improve reasoning with structured thinking
            \item Reiterate key information for better context handling
        \end{itemize}
    \end{itemize}
    
    \begin{center}
        \begin{tikzpicture}
            \node[draw, rounded corners, fill=green!10, text width=9cm, align=center, padding=0.3cm] {
                \textbf{Remember:} Small UX and prompting improvements can dramatically increase effectiveness
            };
        \end{tikzpicture}
    \end{center}
\end{frame} 
% Practical Implementation Section
\newcommand{\practicalimplementationframe}{
    \section{Practical Implementation Strategies}

    \begin{frame}
        \frametitle{Synthetic Data Generation}
        \begin{itemize}
            \item Use LLMs to generate domain-specific questions from your documents
            \item Create diverse query types:
            \begin{itemize}
                \item Factual: "What was the revenue in Q2 2023?"
                \item Comparative: "How did Q2 performance compare to Q1?"
                \item Analytical: "What factors contributed to the margin decline?"
            \end{itemize}
            \item Include edge cases and known failure modes
            \item Example prompt: 
            \begin{quote}
                \small "Generate 10 questions a financial analyst might ask about this earnings report, including questions about revenue trends, profitability metrics, and forward guidance."
            \end{quote}
        \end{itemize}
    \end{frame}

    \begin{frame}
        \frametitle{Evaluation Metrics}
        \begin{itemize}
            \item \textbf{Recall@k}: Percentage of relevant documents in top k results
            \begin{itemize}
                \item Critical for RAG - can't generate from missing information
                \item Target 80-90\% recall before focusing on generation quality
            \end{itemize}
            \item \textbf{MRR (Mean Reciprocal Rank)}: Position of first relevant document
            \begin{itemize}
                \item Higher weight to documents appearing earlier in results
                \item Useful for prioritizing most relevant content
            \end{itemize}
            \item \textbf{Latency}: Response time for retrieval operations
            \begin{itemize}
                \item Critical for user experience and scaling
                \item Balance between accuracy and speed
            \end{itemize}
            \item \textbf{Coverage}: Percentage of query types that can be answered
            \begin{itemize}
                \item Identifies gaps in retrieval capabilities
                \item Helps prioritize new index or tool development
            \end{itemize}
        \end{itemize}
    \end{frame}

    \begin{frame}
        \frametitle{Evaluation Implementation Tips}
        \begin{itemize}
            \item Start with 100-200 test examples across different categories
            \begin{itemize}
                \item 20-30 examples per major query type is sufficient
                \item Include examples from each domain area
            \end{itemize}
            \item Automate evaluation to run in under 1 minute
            \begin{itemize}
                \item Fast feedback loops enable rapid iteration
                \item Run after every significant change
            \end{itemize}
            \item Log all failures for systematic analysis
            \begin{itemize}
                \item Store full context of failed retrievals
                \item Look for patterns in failure modes
                \item Use failures to generate new test cases
            \end{itemize}
        \end{itemize}
    \end{frame}

    \begin{frame}
        \frametitle{Query Segmentation Approaches}
        \begin{itemize}
            \item \textbf{Intent-based}: What is the user trying to accomplish?
            \begin{itemize}
                \item Information seeking vs. task completion
                \item Exploratory vs. targeted queries
                \item Example: "Tell me about X" vs. "How do I do Y?"
            \end{itemize}
            \item \textbf{Domain-based}: Which knowledge area does this touch?
            \begin{itemize}
                \item Subject matter categories
                \item Technical vs. business vs. compliance
                \item Example: Financial metrics vs. operational details
            \end{itemize}
            \item \textbf{Complexity-based}: Simple lookup vs. multi-step reasoning
            \begin{itemize}
                \item Single fact retrieval vs. synthesis across documents
                \item Explicit vs. implicit information
                \item Example: Direct value lookup vs. trend analysis
            \end{itemize}
            \item \textbf{Data-type}: Text, table, image, code, or multi-modal
            \begin{itemize}
                \item Different content formats require different handling
                \item Example: Narrative text vs. tabular financial data
            \end{itemize}
        \end{itemize}
    \end{frame}

    \begin{frame}
        \frametitle{Segmentation Implementation}
        \begin{itemize}
            \item Use a classifier prompt to categorize each query
            \begin{itemize}
                \item Chain-of-thought prompting improves accuracy
                \item Allow multiple categories per query when relevant
            \end{itemize}
            \item Track performance metrics per segment
            \begin{itemize}
                \item Separate dashboards for each major category
                \item Compare performance across segments
            \end{itemize}
            \item Identify segments with highest volume × lowest performance
            \begin{itemize}
                \item Focus improvements on high-impact areas
                \item Prioritize based on business value
            \end{itemize}
        \end{itemize}
    \end{frame}

    \begin{frame}
        \frametitle{Example Segmentation for Financial Projects}
        \begin{itemize}
            \item \textbf{Numerical extraction}
            \begin{itemize}
                \item Finding specific values in financial statements
                \item Example: "What was the EPS in Q2 2023?"
                \item Requires precise table extraction and entity recognition
            \end{itemize}
            \item \textbf{Trend analysis}
            \begin{itemize}
                \item Identifying patterns over time in financial data
                \item Example: "How has the gross margin changed over the last 4 quarters?"
                \item Requires time-series data and comparative analysis
            \end{itemize}
            \item \textbf{Comparative analysis}
            \begin{itemize}
                \item Comparing entities, periods, or metrics
                \item Example: "How does Company X's ROI compare to industry average?"
                \item Requires multi-document retrieval and normalization
            \end{itemize}
            \item \textbf{Risk assessment}
            \begin{itemize}
                \item Evaluating potential issues or concerns
                \item Example: "What are the key risk factors mentioned in the report?"
                \item Requires understanding of risk terminology and context
            \end{itemize}
        \end{itemize}
    \end{frame}

    \begin{frame}
        \frametitle{Hybrid Search Implementation}
        \begin{itemize}
            \item Combine BM25/lexical search with semantic/embedding search
            \begin{itemize}
                \item Lexical: Exact keyword matching (Elasticsearch, BM25)
                \item Semantic: Meaning-based matching (embeddings)
                \item Hybrid: Best of both approaches
            \end{itemize}
            \item Weight results based on query characteristics
            \begin{itemize}
                \item Adjust weights dynamically based on query type
                \item Use query classifier to determine optimal weights
            \end{itemize}
            \item Example implementation:
            \begin{itemize}
                \item Higher weight on lexical for specific terms, codes, IDs
                \item Higher weight on semantic for concepts, themes, topics
                \item Ensemble methods to combine results effectively
            \end{itemize}
        \end{itemize}
    \end{frame}

    \begin{frame}
        \frametitle{Metadata Enhancement}
        \begin{itemize}
            \item Extract structured data during ingestion
            \begin{itemize}
                \item Dates, time periods, fiscal quarters
                \item Entity names, categories, product types
                \item Document types, sections, importance levels
            \end{itemize}
            \item Create filters based on document attributes
            \begin{itemize}
                \item Filter by date range, document type, entity
                \item Combine filters with semantic search
                \item Improve precision without hurting recall
            \end{itemize}
            \item Enable faceted search capabilities
            \begin{itemize}
                \item Allow users to narrow results by metadata
                \item Provide context-aware filtering options
                \item Improve user experience and control
            \end{itemize}
        \end{itemize}
    \end{frame}

    \begin{frame}
        \frametitle{Specialized Indices}
        \begin{itemize}
            \item Create separate indices for different content types
            \begin{itemize}
                \item Text documents (reports, articles, narratives)
                \item Tabular data (financial statements, metrics)
                \item Images and diagrams (charts, graphs, visuals)
                \item Code and structured data (SQL, JSON, XML)
            \end{itemize}
            \item Optimize chunking strategy per content type
            \begin{itemize}
                \item Text: Semantic paragraphs or fixed-size chunks
                \item Tables: Preserve headers and context
                \item Code: Function or class-level chunks
            \end{itemize}
            \item Example specialized indices:
            \begin{itemize}
                \item Table index with header preservation
                \item Text index with paragraph-level chunking
                \item Entity index for quick lookups
                \item Time-series index for historical data
            \end{itemize}
        \end{itemize}
    \end{frame}

    \begin{frame}
        \frametitle{Tool-Based Approach}
        \begin{itemize}
            \item Define clear interfaces for each specialized retrieval method
            \begin{itemize}
                \item Consistent input/output formats
                \item Well-defined parameters and options
                \item Clear documentation and examples
            \end{itemize}
            \item Create a router that selects appropriate tool(s) for each query
            \begin{itemize}
                \item Use LLM to classify and route queries
                \item Consider confidence scores for tool selection
                \item Fall back to general search when uncertain
            \end{itemize}
            \item Consider parallel execution for better performance
            \begin{itemize}
                \item Run multiple tools simultaneously when appropriate
                \item Merge results with intelligent ranking
                \item Balance latency vs. thoroughness
            \end{itemize}
        \end{itemize}
    \end{frame}

    \begin{frame}
        \frametitle{Query Routing Implementation}
        \begin{itemize}
            \item Start with 3-5 core tools based on your domain needs
            \begin{itemize}
                \item Begin with the most common query types
                \item Add specialized tools as patterns emerge
                \item Keep interfaces consistent across tools
            \end{itemize}
            \item Test routing accuracy with synthetic queries
            \begin{itemize}
                \item Create test cases for each tool
                \item Measure routing precision and recall
                \item Identify confusion patterns between tools
            \end{itemize}
            \item Measure tool recall: Is the right tool being selected?
            \begin{itemize}
                \item Track correct tool selection rate
                \item Monitor unnecessary tool calls
                \item Improve router prompts based on errors
            \end{itemize}
        \end{itemize}
    \end{frame}

    \begin{frame}
        \frametitle{Example Tools for Financial Domain}
        \begin{itemize}
            \item \texttt{table\_search}: Find and extract tabular financial data
            \begin{itemize}
                \item Specialized for financial statements, tables, metrics
                \item Preserves row/column relationships
                \item Example: "What was the operating margin in Q2?"
            \end{itemize}
            \item \texttt{text\_search}: Retrieve narrative sections of reports
            \begin{itemize}
                \item Optimized for MD\&A, risk factors, footnotes
                \item Semantic paragraph-level retrieval
                \item Example: "Explain the factors affecting revenue growth"
            \end{itemize}
            \item \texttt{entity\_lookup}: Find information about specific companies
            \begin{itemize}
                \item Company profiles, key metrics, industry data
                \item Fast retrieval of entity-specific information
                \item Example: "What is Company X's market position?"
            \end{itemize}
            \item \texttt{time\_series}: Retrieve historical financial metrics
            \begin{itemize}
                \item Optimized for trend analysis and comparisons
                \item Handles time-based queries efficiently
                \item Example: "Show revenue growth over the last 8 quarters"
            \end{itemize}
        \end{itemize}
    \end{frame}

    \begin{frame}
        \frametitle{Effective Feedback Mechanisms}
        \begin{itemize}
            \item Thumbs up/down with reason categorization
            \begin{itemize}
                \item Simple binary feedback is better than none
                \item Add reason categories for negative feedback
                \item Example categories: Irrelevant, Incomplete, Incorrect, Outdated
            \end{itemize}
            \item Citation validation
            \begin{itemize}
                \item "Were these sources helpful?" for each citation
                \item Allow users to mark irrelevant citations
                \item Track which sources are most valuable
            \end{itemize}
            \item Implicit signals
            \begin{itemize}
                \item Time spent reviewing results
                \item Follow-up questions or refinements
                \item Copied content or saved responses
                \item Repeated queries indicating dissatisfaction
            \end{itemize}
        \end{itemize}
    \end{frame}

    \begin{frame}
        \frametitle{Feedback Implementation Tips}
        \begin{itemize}
            \item Make feedback UI prominent and simple
            \begin{itemize}
                \item Large, obvious buttons increase engagement
                \item Modal dialogs get 4-5x more feedback than subtle buttons
                \item Keep the initial feedback step quick (1-2 clicks)
            \end{itemize}
            \item Categorize negative feedback by failure type
            \begin{itemize}
                \item Retrieval failures vs. generation issues
                \item Missing information vs. incorrect information
                \item Technical errors vs. content gaps
            \end{itemize}
            \item Store feedback with full context for analysis
            \begin{itemize}
                \item Original query and retrieved documents
                \item Generated response and citations
                \item User feedback and any follow-up actions
                \item System metadata (latency, model version, etc.)
            \end{itemize}
        \end{itemize}
    \end{frame}

    \begin{frame}
        \frametitle{Continuous Improvement Cycle}
        \begin{itemize}
            \item Weekly review of feedback patterns
            \begin{itemize}
                \item Analyze trends in negative feedback
                \item Identify common failure modes
                \item Track improvement over time
            \end{itemize}
            \item Prioritize fixes based on frequency and impact
            \begin{itemize}
                \item Focus on high-volume, high-impact issues first
                \item Balance quick wins vs. structural improvements
                \item Create targeted test cases for each issue
            \end{itemize}
            \item Use feedback to generate new test cases
            \begin{itemize}
                \item Convert real failures into synthetic test examples
                \item Expand test coverage based on user behavior
                \item Validate fixes against expanded test set
            \end{itemize}
            \item Retrain/fine-tune models with collected data
            \begin{itemize}
                \item Use feedback to create training pairs
                \item Fine-tune embeddings or rerankers
                \item Improve router accuracy with real examples
            \end{itemize}
        \end{itemize}
    \end{frame}

    \begin{frame}[fragile]
        \frametitle{Code Example: Simple Evaluation Pipeline}
        \begin{lstlisting}[basicstyle=\tiny\ttfamily, breaklines=true]
def evaluate_retrieval(test_queries, retriever, k=5):
    results = {}
    for query_id, query in test_queries.items():
        # Get ground truth documents
        relevant_docs = query["relevant_docs"]
        
        # Retrieve documents
        retrieved_docs = retriever.search(
            query["text"], k=k
        )
        
        # Calculate metrics
        recall = len(set(retrieved_docs) & 
                     set(relevant_docs)) / len(relevant_docs)
        
        results[query_id] = {
            "query": query["text"],
            "recall": recall,
            "retrieved": retrieved_docs,
            "relevant": relevant_docs
        }
    
    # Aggregate metrics
    avg_recall = sum(r["recall"] for r in 
                    results.values()) / len(results)
    
    return avg_recall, results
        \end{lstlisting}
    \end{frame}

    \begin{frame}
        \frametitle{Implementation Roadmap: Weeks 1-4}
        \begin{itemize}
            \item \textbf{Week 1-2: Evaluation Foundation}
            \begin{itemize}
                \item Generate synthetic test data for your domain
                \item Implement basic retrieval evaluation pipeline
                \item Establish baseline metrics for recall and precision
                \item Create initial dashboard for tracking performance
            \end{itemize}
            \item \textbf{Week 3-4: Segmentation \& Analysis}
            \begin{itemize}
                \item Define query categories relevant to your domain
                \item Classify existing queries into segments
                \item Analyze performance by segment
                \item Identify priority improvement areas
                \item Create segment-specific test sets
            \end{itemize}
        \end{itemize}
    \end{frame}

    \begin{frame}
        \frametitle{Implementation Roadmap: Weeks 5-8}
        \begin{itemize}
            \item \textbf{Week 5-6: Specialized Retrieval}
            \begin{itemize}
                \item Implement domain-specific indices
                \item Build basic query router
                \item Test with synthetic and real queries
                \item Optimize chunking strategies
                \item Implement metadata filters
            \end{itemize}
            \item \textbf{Week 7-8: Feedback \& Refinement}
            \begin{itemize}
                \item Add feedback collection mechanisms
                \item Analyze initial feedback
                \item Begin fine-tuning based on results
                \item Implement continuous improvement process
                \item Set up regular review cadence
            \end{itemize}
        \end{itemize}
    \end{frame}
} % End of practicalimplementationframe command 
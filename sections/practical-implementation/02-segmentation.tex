% Query Segmentation Section

\begin{frame}
    \frametitle{Segmentation: The Foundation for Improvement}
    
    \begin{center}
        \begin{tikzpicture}
            \node[draw, rounded corners, fill=blue!5, text width=9cm, align=center, padding=0.3cm] {
                \textbf{Why segment queries?} To identify specific areas for improvement
            };
        \end{tikzpicture}
    \end{center}
    
    \begin{itemize}
        \item Moving beyond basic metrics (recall/precision)
        \item Identifying patterns in user behavior
        \item Prioritizing development efforts based on data
        \item Tracking performance across different query types
    \end{itemize}
    
    \begin{center}
        \begin{tikzpicture}
            \node[draw, rounded corners, fill=yellow!10, text width=8.5cm, align=center, padding=0.3cm] {
                \textbf{Key insight:} Summary statistics often mask important patterns
            };
        \end{tikzpicture}
    \end{center}
\end{frame}

% Query Segmentation Approaches - Split into two slides
\begin{frame}
    \frametitle{Query Segmentation Approaches (1/2)}
    \begin{itemize}
        \item \textbf{Intent-based}: What is the user trying to accomplish?
        \begin{itemize}
            \item Information seeking vs. task completion
            \item Exploratory vs. targeted queries
            \item Example: "Tell me about X" vs. "How do I do Y?"
        \end{itemize}
        \item \textbf{Domain-based}: Which knowledge area does this touch?
        \begin{itemize}
            \item Subject matter categories
            \item Technical vs. business vs. compliance
            \item Example: Financial metrics vs. operational details
        \end{itemize}
    \end{itemize}
\end{frame}

\begin{frame}
    \frametitle{Query Segmentation Approaches (2/2)}
    \begin{itemize}
        \item \textbf{Complexity-based}: Simple lookup vs. multi-step reasoning
        \begin{itemize}
            \item Single fact retrieval vs. synthesis across documents
            \item Explicit vs. implicit information
            \item Example: Direct value lookup vs. trend analysis
        \end{itemize}
        \item \textbf{Data-type}: Text, table, image, code, or multi-modal
        \begin{itemize}
            \item Different content formats require different handling
            \item Example: Narrative text vs. tabular financial data
        \end{itemize}
    \end{itemize}
\end{frame}

% Inventory vs. Capabilities - New section based on week-2.txt
\begin{frame}
    \frametitle{Categorizing Issues: Inventory vs. Capabilities}
    
    \begin{columns}
        \column{0.48\textwidth}
        \textbf{Inventory Issues}
        \begin{itemize}
            \item Missing content in knowledge base
            \item Incomplete data sources
            \item Outdated information
            \item Gaps in document coverage
        \end{itemize}
        
        \column{0.48\textwidth}
        \textbf{Capability Issues}
        \begin{itemize}
            \item System's functional limitations
            \item Missing metadata extraction
            \item Lack of structured filtering
            \item Insufficient query understanding
        \end{itemize}
    \end{columns}
    
    \begin{center}
        \begin{tikzpicture}
            \node[draw, rounded corners, fill=green!5, text width=8.5cm, align=center, padding=0.3cm] {
                \textbf{Different solutions:} Inventory → expand corpus\\
                Capabilities → build new functionality
            };
        \end{tikzpicture}
    \end{center}
\end{frame}

\begin{frame}
    \frametitle{Detecting Inventory vs. Capability Gaps}
    
    \begin{columns}
        \column{0.48\textwidth}
        \textbf{Inventory Gap Signals}
        \begin{itemize}
            \item Low cosine similarities
            \item No results from lexical search
            \item LLM refusing to answer
            \item Returned chunks never cited
            \item Broken data pipelines
        \end{itemize}
        
        \column{0.48\textwidth}
        \textbf{Capability Gap Examples}
        \begin{itemize}
            \item Time-based filtering needs
            \item Comparison across documents
            \item Structured data extraction
            \item Missing metadata fields
            \item Need for specialized indices
        \end{itemize}
    \end{columns}
    
    \begin{center}
        \begin{tikzpicture}
            \node[draw, rounded corners, fill=blue!10, text width=8.5cm, align=center, padding=0.3cm] {
                \textbf{Example:} "Latest contract modifications" requires both\\
                inventory (recent docs) and capability (time filtering)
            };
        \end{tikzpicture}
    \end{center}
\end{frame}

% Segmentation Implementation - Split into smaller parts
\begin{frame}
    \frametitle{Implementing Query Classification}
    \begin{itemize}
        \item Use a classifier prompt to categorize each query
        \begin{itemize}
            \item Chain-of-thought prompting improves accuracy
            \item Allow multiple categories per query when relevant
        \end{itemize}
    \end{itemize}
    
    \vspace{0.5cm}
    
    \begin{center}
        \begin{tikzpicture}
            \node[draw, rounded corners, fill=blue!10, text width=8cm, align=center, padding=0.3cm] {
                \textbf{Prompt Example:}\\
                "Analyze this query and determine which category it belongs to. Explain your reasoning before giving your final answer."
            };
        \end{tikzpicture}
    \end{center}
\end{frame}

\begin{frame}
    \frametitle{Tracking Segmentation Performance}
    \begin{itemize}
        \item Track performance metrics per segment
        \begin{itemize}
            \item Separate dashboards for each major category
            \item Compare performance across segments
        \end{itemize}
        \item Identify segments with highest volume × lowest performance
        \begin{itemize}
            \item Focus improvements on high-impact areas
            \item Prioritize based on business value
        \end{itemize}
    \end{itemize}
\end{frame}

\begin{frame}
    \frametitle{Prioritization Framework}
    
    \begin{columns}
        \column{0.6\textwidth}
        \textbf{Prioritization Formula}
        \begin{itemize}
            \item Impact of answering this type of question
            \item Query volume for the segment
            \item Likelihood of success (can we solve it?)
        \end{itemize}
        
        \column{0.4\textwidth}
        \begin{center}
            \begin{tikzpicture}
                \node[draw, rounded corners, fill=yellow!10, text width=4cm, align=center, padding=0.3cm] {
                    Priority = Impact × Volume × P(Success)
                };
            \end{tikzpicture}
        \end{center}
    \end{columns}
    
    \vspace{0.3cm}
    \textbf{Decision Matrix}
    \begin{itemize}
        \item High satisfaction + High volume = Maintain
        \item High satisfaction + Low volume = Promote
        \item Low satisfaction + Low volume = Phase out
        \item Low satisfaction + High volume = Focus here!
    \end{itemize}
\end{frame}

% Key Insights for Week 2
\begin{frame}
    \frametitle{Week 2: Key Insights}
    
    \begin{center}
        \begin{tikzpicture}
            \node[draw, rounded corners, fill=blue!5, text width=9cm, align=center, padding=0.3cm] {
                \textbf{Query Segmentation: The Path to Targeted Improvements}
            };
        \end{tikzpicture}
    \end{center}
    
    \begin{itemize}
        \item \textbf{Segment queries} by intent, domain, complexity, and data type
        \item \textbf{Distinguish between inventory and capability gaps}
            \begin{itemize}
                \item Inventory: Missing content → Add more data
                \item Capability: System limitations → Build new features
            \end{itemize}
        \item \textbf{Track performance by segment} to identify specific weaknesses
        \item \textbf{Prioritize improvements} based on impact, volume, and feasibility
        \item \textbf{Focus on high-volume, low-satisfaction segments} first
    \end{itemize}
    
    \begin{center}
        \begin{tikzpicture}
            \node[draw, rounded corners, fill=green!10, text width=9cm, align=center, padding=0.3cm] {
                \textbf{Remember:} Summary statistics often mask important patterns
            };
        \end{tikzpicture}
    \end{center}
\end{frame}

% Example Segmentation for Financial Projects - Split into two slides
\begin{frame}
    \frametitle{Financial Query Types (1/2)}
    \begin{itemize}
        \item \textbf{Numerical extraction}
        \begin{itemize}
            \item Finding specific values in financial statements
            \item Example: "What was the EPS in Q2 2023?"
            \item Requires precise table extraction and entity recognition
        \end{itemize}
        \item \textbf{Trend analysis}
        \begin{itemize}
            \item Identifying patterns over time in financial data
            \item Example: "How has the gross margin changed over the last 4 quarters?"
            \item Requires time-series data and comparative analysis
        \end{itemize}
    \end{itemize}
\end{frame}

\begin{frame}
    \frametitle{Financial Query Types (2/2)}
    \begin{itemize}
        \item \textbf{Comparative analysis}
        \begin{itemize}
            \item Comparing entities, periods, or metrics
            \item Example: "How does Company X's ROI compare to industry average?"
            \item Requires multi-document retrieval and normalization
        \end{itemize}
        \item \textbf{Risk assessment}
        \begin{itemize}
            \item Evaluating potential issues or concerns
            \item Example: "What are the key risk factors mentioned in the report?"
            \item Requires understanding of risk terminology and context
        \end{itemize}
    \end{itemize}
\end{frame}

\begin{frame}
    \frametitle{Real-World Example: Construction Project}
    
    \begin{columns}
        \column{0.48\textwidth}
        \textbf{Initial Analysis}
        \begin{itemize}
            \item 80% of queries were document search
            \item 20% were schedule-based queries
            \item Document search had high satisfaction
            \item Schedule queries had low satisfaction
        \end{itemize}
        
        \column{0.48\textwidth}
        \textbf{Time-Based Analysis}
        \begin{itemize}
            \item New users started with schedule queries
            \item Poor results led to behavior change
            \item Users learned to use document search as workaround
            \item Masked the actual problem
        \end{itemize}
    \end{columns}
    
    \vspace{0.3cm}
    \begin{center}
        \begin{tikzpicture}
            \node[draw, rounded corners, fill=blue!10, text width=8.5cm, align=center, padding=0.3cm] {
                \textbf{Solution:} Built specialized data extraction for dates and schedules\\
                Highlighted new capability → user behavior changed back
            };
        \end{tikzpicture}
    \end{center}
\end{frame}

% Case Study: Scheduling & Learned User Behavior
\begin{frame}
    \frametitle{Case Study: Scheduling \& Learned User Behavior}
    
    \begin{columns}
        \column{0.48\textwidth}
        \textbf{The Challenge}
        \begin{itemize}
            \item Construction management platform
            \item New users struggled with scheduling queries
            \item Veteran users learned workarounds
            \item Only 20\% success rate for new users on scheduling queries
        \end{itemize}
        
        \column{0.48\textwidth}
        \textbf{The Solution}
        \begin{itemize}
            \item Split "document search" vs. "scheduling" segments
            \item Created specialized scheduling index
            \item Extracted due dates and milestones
            \item Announced new feature to users
        \end{itemize}
    \end{columns}
    
    \vspace{0.3cm}
    \begin{center}
        \begin{tikzpicture}
            \node[draw, rounded corners, fill=blue!5, text width=9cm, align=center, padding=0.3cm] {
                \textbf{Key Takeaway:} Segmentation + dedicated metadata indices dramatically improve user satisfaction
            };
        \end{tikzpicture}
    \end{center}
\end{frame} 
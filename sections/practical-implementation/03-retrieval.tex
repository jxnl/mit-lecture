% Specialized Retrieval Methods Section

% Hybrid Search Implementation - Split into two slides
\begin{frame}
    \frametitle{Hybrid Search: Combining Approaches}
    \begin{columns}
        \column{0.5\textwidth}
        \textbf{Lexical Search}
        \begin{itemize}
            \item Exact keyword matching
            \item Based on BM25, TF-IDF
            \item Great for precise terms
            \item Example: Elasticsearch
        \end{itemize}
        
        \column{0.5\textwidth}
        \textbf{Semantic Search}
        \begin{itemize}
            \item Meaning-based matching
            \item Uses embeddings
            \item Great for concepts
            \item Handles synonyms
        \end{itemize}
    \end{columns}
    
    \vspace{0.5cm}
    \begin{center}
        \begin{tikzpicture}
            \node[draw, rounded corners, fill=blue!10, text width=8cm, align=center, padding=0.3cm] {
                \textbf{Hybrid Search:} Combines strengths of both approaches
            };
        \end{tikzpicture}
    \end{center}
\end{frame}

\begin{frame}
    \frametitle{Hybrid Search: Implementation}
    \begin{itemize}
        \item Weight results based on query characteristics
        \begin{itemize}
            \item Adjust weights dynamically based on query type
            \item Use query classifier to determine optimal weights
        \end{itemize}
    \end{itemize}
    
    \vspace{0.3cm}
    \textbf{When to favor each approach:}
    \begin{itemize}
        \item \textbf{Lexical}: Specific terms, codes, IDs, exact phrases
        \item \textbf{Semantic}: Concepts, themes, topics, intentions
    \end{itemize}
    
    \vspace{0.3cm}
    \begin{center}
        \begin{tikzpicture}
            \node[draw, rounded corners, fill=green!5, text width=8.5cm, align=center, padding=0.3cm] {
                \textbf{Example:} "Q2 2023 revenue" → 70% lexical, 30% semantic\\
                "Growth strategy reasons" → 20% lexical, 80% semantic
            };
        \end{tikzpicture}
    \end{center}
\end{frame}

% Metadata Enhancement - Split into two slides
\begin{frame}
    \frametitle{Metadata Enhancement: Extraction}
    \begin{itemize}
        \item Extract structured data during ingestion
    \end{itemize}
    
    \begin{columns}
        \column{0.33\textwidth}
        \textbf{Temporal}
        \begin{itemize}
            \item Dates
            \item Time periods
            \item Fiscal quarters
            \item Years
        \end{itemize}
        
        \column{0.33\textwidth}
        \textbf{Entities}
        \begin{itemize}
            \item Companies
            \item People
            \item Products
            \item Categories
        \end{itemize}
        
        \column{0.33\textwidth}
        \textbf{Document}
        \begin{itemize}
            \item Doc type
            \item Sections
            \item Importance
            \item Source
        \end{itemize}
    \end{columns}
    
    \vspace{0.3cm}
    \begin{center}
        \begin{tikzpicture}
            \node[draw, rounded corners, fill=yellow!10, text width=8cm, align=center, padding=0.3cm] {
                \textbf{Key insight:} Rich metadata enables powerful filtering
            };
        \end{tikzpicture}
    \end{center}
\end{frame}

\begin{frame}
    \frametitle{Metadata Enhancement: Filtering}
    \begin{itemize}
        \item Create filters based on document attributes
        \begin{itemize}
            \item Filter by date range, document type, entity
            \item Combine filters with semantic search
            \item Improve precision without hurting recall
        \end{itemize}
        \item Enable faceted search capabilities
        \begin{itemize}
            \item Allow users to narrow results by metadata
            \item Provide context-aware filtering options
        \end{itemize}
    \end{itemize}
    
    \begin{center}
        \begin{tikzpicture}
            \node[draw, rounded corners, fill=blue!10, text width=8.5cm, align=center, padding=0.3cm] {
                \textbf{Example:} "Revenue for Q2 2023 in North America region"\\
                \texttt{embedding\_search(query) AND date="Q2 2023" AND region="NA"}
            };
        \end{tikzpicture}
    \end{center}
\end{frame}

% Specialized Indices - Split into two slides
\begin{frame}
    \frametitle{Specialized Indices: Content Types}
    \begin{itemize}
        \item Create separate indices for different content types
    \end{itemize}
    
    \begin{columns}
        \column{0.5\textwidth}
        \textbf{Text Documents}
        \begin{itemize}
            \item Reports
            \item Articles
            \item Narratives
            \item Analysis
        \end{itemize}
        
        \textbf{Tabular Data}
        \begin{itemize}
            \item Financial statements
            \item Metrics tables
            \item KPI dashboards
        \end{itemize}
        
        \column{0.5\textwidth}
        \textbf{Visual Content}
        \begin{itemize}
            \item Charts
            \item Graphs
            \item Diagrams
        \end{itemize}
        
        \textbf{Structured Data}
        \begin{itemize}
            \item SQL
            \item JSON
            \item XML
            \item Code
        \end{itemize}
    \end{columns}
\end{frame}

\begin{frame}
    \frametitle{Specialized Indices: Chunking Strategies}
    \begin{itemize}
        \item Optimize chunking strategy per content type
    \end{itemize}
    
    \begin{columns}
        \column{0.5\textwidth}
        \textbf{Text}
        \begin{itemize}
            \item Semantic paragraphs
            \item Fixed-size chunks
            \item Sliding window
        \end{itemize}
        
        \textbf{Tables}
        \begin{itemize}
            \item Preserve headers
            \item Keep context
            \item Include table titles
        \end{itemize}
        
        \column{0.5\textwidth}
        \textbf{Code}
        \begin{itemize}
            \item Function-level chunks
            \item Class-level chunks
            \item Keep imports
        \end{itemize}
        
        \textbf{Specialized}
        \begin{itemize}
            \item Entity index
            \item Time-series index
            \item KPI index
        \end{itemize}
    \end{columns}
\end{frame}

% Tool-Based Approach - Split into two slides
\begin{frame}
    \frametitle{Tool-Based Approach: Interfaces}
    
    \begin{center}
        \begin{tikzpicture}
            \node[draw, rounded corners, fill=blue!5, text width=9cm, align=center, padding=0.3cm] {
                \textbf{Key concept:} Define specialized search tools with clear interfaces
            };
        \end{tikzpicture}
    \end{center}
    
    \vspace{0.3cm}
    \textbf{Tool Design Principles}
    \begin{itemize}
        \item Define clear interfaces for each specialized retrieval method
        \begin{itemize}
            \item Consistent input/output formats
            \item Well-defined parameters and options
            \item Clear documentation and examples
        \end{itemize}
    \end{itemize}
    
    \vspace{0.3cm}
    \begin{center}
        \begin{tikzpicture}
            \node[draw, rounded corners, fill=green!5, text width=8.5cm, align=center, padding=0.3cm] {
                \textbf{Example Interface:}\\
                \texttt{table\_search(query, date\_range, metrics=["revenue", "margin"])}
            };
        \end{tikzpicture}
    \end{center}
\end{frame}

% Key Insights for Week 3
\begin{frame}
    \frametitle{Week 3: Key Insights}
    
    \begin{center}
        \begin{tikzpicture}
            \node[draw, rounded corners, fill=blue!5, text width=9cm, align=center, padding=0.3cm] {
                \textbf{Specialized Retrieval: Beyond Basic Search}
            };
        \end{tikzpicture}
    \end{center}
    
    \begin{itemize}
        \item \textbf{Implement hybrid search} combining lexical and semantic approaches
        \begin{itemize}
            \item Lexical: Exact matches, codes, IDs, specific terms
            \item Semantic: Concepts, themes, intentions, synonyms
        \end{itemize}
        \item \textbf{Extract rich metadata} during document ingestion
        \item \textbf{Create specialized indices} for different content types
        \begin{itemize}
            \item Text, tables, code, images each need different handling
        \end{itemize}
        \item \textbf{Design tool-based interfaces} for specialized retrieval methods
    \end{itemize}
    
    \begin{center}
        \begin{tikzpicture}
            \node[draw, rounded corners, fill=green!10, text width=9cm, align=center, padding=0.3cm] {
                \textbf{Remember:} Different content types require different retrieval strategies
            };
        \end{tikzpicture}
    \end{center}
\end{frame}

\begin{frame}
    \frametitle{Tool-Based Approach: Routing}
    \textbf{Query Router}
    \begin{itemize}
        \item Create a router that selects appropriate tool(s) for each query
        \begin{itemize}
            \item Use LLM to classify and route queries
            \item Consider confidence scores for tool selection
            \item Fall back to general search when uncertain
        \end{itemize}
    \end{itemize}
    
    \vspace{0.3cm}
    \textbf{Execution Strategy}
    \begin{itemize}
        \item Consider parallel execution for better performance
        \begin{itemize}
            \item Run multiple tools simultaneously when appropriate
            \item Merge results with intelligent ranking
            \item Balance latency vs. thoroughness
        \end{itemize}
    \end{itemize}
\end{frame}

% Query Routing Implementation - Split into two slides
\begin{frame}
    \frametitle{Implementing Query Routing}
    \begin{columns}
        \column{0.5\textwidth}
        \textbf{Start Simple}
        \begin{itemize}
            \item Begin with 3-5 core tools
            \item Focus on common query types
            \item Add more as patterns emerge
        \end{itemize}
        
        \column{0.5\textwidth}
        \textbf{Consistency Matters}
        \begin{itemize}
            \item Standardize tool interfaces
            \item Common parameter formats
            \item Predictable outputs
        \end{itemize}
    \end{columns}
    
    \vspace{0.3cm}
    \begin{center}
        \begin{tikzpicture}
            \node[draw, rounded corners, fill=yellow!10, text width=8.5cm, align=center, padding=0.3cm] {
                \textbf{Testing tip:} Create synthetic queries for each tool to validate router accuracy
            };
        \end{tikzpicture}
    \end{center}
\end{frame}

\begin{frame}
    \frametitle{Measuring Router Performance}
    \begin{itemize}
        \item Test routing accuracy with synthetic queries
        \begin{itemize}
            \item Create test cases for each tool
            \item Measure routing precision and recall
            \item Identify confusion patterns between tools
        \end{itemize}
        \item Measure tool recall: Is the right tool being selected?
        \begin{itemize}
            \item Track correct tool selection rate
            \item Monitor unnecessary tool calls
            \item Improve router prompts based on errors
        \end{itemize}
    \end{itemize}
\end{frame}

% Example Tools for Financial Domain - Split into two slides
\begin{frame}
    \frametitle{Financial Domain Tools (1/2)}
    
    \begin{columns}
        \column{0.5\textwidth}
        \textbf{\texttt{table\_search}}
        \begin{itemize}
            \item Finds financial tables
            \item Preserves row/column context
            \item Extracts precise metrics
            \item Example: "Q2 operating margin"
        \end{itemize}
        
        \column{0.5\textwidth}
        \textbf{\texttt{text\_search}}
        \begin{itemize}
            \item Retrieves narratives
            \item MD\&A, risk factors
            \item Semantic paragraph search
            \item Example: "Revenue growth factors"
        \end{itemize}
    \end{columns}
\end{frame}

\begin{frame}
    \frametitle{Financial Domain Tools (2/2)}
    
    \begin{columns}
        \column{0.5\textwidth}
        \textbf{\texttt{entity\_lookup}}
        \begin{itemize}
            \item Company profiles
            \item Key metrics
            \item Industry data
            \item Example: "Company X's position"
        \end{itemize}
        
        \column{0.5\textwidth}
        \textbf{\texttt{time\_series}}
        \begin{itemize}
            \item Historical metrics
            \item Trend analysis
            \item Period comparisons
            \item Example: "Revenue growth over 8 quarters"
        \end{itemize}
    \end{columns}
    
    \vspace{0.3cm}
    \begin{center}
        \begin{tikzpicture}
            \node[draw, rounded corners, fill=blue!10, text width=8.5cm, align=center, padding=0.3cm] {
                \textbf{Start with tools that address your most common queries}
            };
        \end{tikzpicture}
    \end{center}
\end{frame}

% Case Study: Construction Blueprints & Visual Summaries
\begin{frame}
    \frametitle{Case Study: Construction Blueprints \& Visual Summaries}
    
    \begin{columns}
        \column{0.48\textwidth}
        \textbf{The Challenge}
        \begin{itemize}
            \item Construction company needed AI to answer blueprint questions
            \item Simple image captioning gave generic descriptions
            \item Only ~27\% recall on specific blueprint questions
            \item Multimodal retrieval was ineffective
        \end{itemize}
        
        \column{0.48\textwidth}
        \textbf{The Solution}
        \begin{itemize}
            \item Prompted for detailed descriptions
            \item "Count floors, label mechanical rooms, highlight windows"
            \item Added specialized bounding-box extraction
            \item Merged text + blueprint data
        \end{itemize}
    \end{columns}
    
    \vspace{0.3cm}
    \begin{center}
        \begin{tikzpicture}
            \node[draw, rounded corners, fill=yellow!10, text width=9cm, align=center, padding=0.3cm] {
                \textbf{Key Takeaway:} Specific extraction prompts improved recall from 27\% to 75-85\% in just days
            };
        \end{tikzpicture}
    \end{center}
\end{frame}

% Key Insights for Week 4
\begin{frame}
    \frametitle{Week 4: Key Insights}
    
    \begin{center}
        \begin{tikzpicture}
            \node[draw, rounded corners, fill=blue!5, text width=9cm, align=center, padding=0.3cm] {
                \textbf{Query Routing: Directing Queries to the Right Tools}
            };
        \end{tikzpicture}
    \end{center}
    
    \begin{itemize}
        \item \textbf{Implement a query router} to direct questions to specialized tools
        \begin{itemize}
            \item Use chain-of-thought prompting for better classification
            \item Consider multi-tool routing for complex queries
        \end{itemize}
        \item \textbf{Design clear tool interfaces} with consistent parameters
        \item \textbf{Balance precision and recall} in routing decisions
        \item \textbf{Implement fallback mechanisms} for uncertain classifications
        \item \textbf{Track routing accuracy} as a key performance metric
    \end{itemize}
    
    \begin{center}
        \begin{tikzpicture}
            \node[draw, rounded corners, fill=green!10, text width=9cm, align=center, padding=0.3cm] {
                \textbf{Remember:} The right tool for the right job dramatically improves results
            };
        \end{tikzpicture}
    \end{center}
\end{frame}

% Key Insights for Week 5
\begin{frame}
    \frametitle{Week 5: Key Insights}
    
    \begin{center}
        \begin{tikzpicture}
            \node[draw, rounded corners, fill=blue!5, text width=9cm, align=center, padding=0.3cm] {
                \textbf{Embeddings \& Reranking: Optimizing Relevance}
            };
        \end{tikzpicture}
    \end{center}
    
    \begin{itemize}
        \item \textbf{Choose embedding models} based on domain and query types
        \begin{itemize}
            \item General models for broad topics
            \item Domain-specific models for specialized fields
        \end{itemize}
        \item \textbf{Implement reranking} to improve precision
        \begin{itemize}
            \item Cross-encoders for higher accuracy
            \item LLM-based reranking for complex relevance judgments
        \end{itemize}
        \item \textbf{Fine-tune embeddings} with domain-specific feedback
        \item \textbf{Optimize chunk size} for your specific content and queries
    \end{itemize}
    
    \begin{center}
        \begin{tikzpicture}
            \node[draw, rounded corners, fill=green!10, text width=9cm, align=center, padding=0.3cm] {
                \textbf{Remember:} Better embeddings and reranking can dramatically improve relevance
            };
        \end{tikzpicture}
    \end{center}
\end{frame} 
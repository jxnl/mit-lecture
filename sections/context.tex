% Lecture Context Section
\newcommand{\contextframe}{
    \section{What are we doing here?}

    \begin{frame}
        \frametitle{Course Context \& Goals}
        \begin{itemize}
            \item Learning Objectives
            \begin{itemize}
                \item Developing durable AI knowledge that outlasts specific technical implementations
                \item Understanding ML systems as continuously evolving products rather than "deploy once and forget"
                \item Recognizing the parallels between recommendation systems and retrieval systems
                \item Identifying valuable business applications through effective data analysis
                \item Mastering the key skills for building successful AI systems in the era of democratized tools
            \end{itemize}
        \end{itemize}
    \end{frame}

    \begin{frame}
        \frametitle{Setting the Context}
        \begin{itemize}
            \item Why This Matters Now
            \begin{itemize}
                \item Democratization of AI tools means the competitive advantage comes from thinking deeply
                \item Growing gap between research capabilities and business implementation
                \item Science now drives product development, reversing traditional patterns
                \item Opportunity for individual contributors to have outsized impact through thoughtful implementation
            \end{itemize}
            \item Interactive Format
            \begin{itemize}
                \item This group is quite diverse, so I'll try to keep it broad
                \item The goal is to seed you with good questions, rather than dump information
                \item We'll leave plenty of time for questions about AI, Business, and Career paths
            \end{itemize}
        \end{itemize}
    \end{frame}

    \begin{frame}
        \frametitle{Key Questions to Consider}
        \begin{itemize}
            \item How has machine learning research and implementation evolved from 2015 to 2025?
            \item What behavioral practices should teams adopt when working with AI systems?
            \item How do we identify economically valuable AI applications?
            \item What's the right balance between research exploration and product implementation?
            \item How do we design systems that can evolve effectively over time?
            \item When should you join established labs versus work independently?
            \item How can individuals and small teams achieve leverage without large resources?
            \item What skills matter most in the AI era? (Hint: thinking > coding)
        \end{itemize}
    \end{frame}
    
    \begin{frame}
        \frametitle{Three Key Arcs We'll Explore}
        \begin{itemize}
            \item Technical: From Recommendation to Retrieval Systems
            \begin{itemize}
                \item The surprising similarities in architecture and challenges
                \item Why understanding these parallels helps build better systems
            \end{itemize}
            \item Organizational: Effective AI Implementation
            \begin{itemize}
                \item The importance of observability and measurement
                \item Balancing unified systems vs. specialized subsystems
            \end{itemize}
            \item Personal: Career Considerations in AI
            \begin{itemize}
                \item Information synthesis as a durable skill
                \item How AI changes team dynamics and individual contributions
            \end{itemize}
        \end{itemize}
    \end{frame}
} % End of contextframe command 